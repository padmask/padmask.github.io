%% This is an example first chapter.  You should put chapter/appendix that you
%% write into a separate file, and add a line \include{yourfilename} to
%% main.tex, where `yourfilename.tex' is the name of the chapter/appendix file.
%% You can process specific files by typing their names in at the 
%% \files=
%% prompt when you run the file main.tex through LaTeX.
\chapter{Introduction}
Let $R$ be a discrete valuation ring with fraction field $K$ and residue field $k$. Let $X$ be a {\color{blue}{\textsf{nice}}} (smooth, projective and geometrically integral) $K$-variety. The variety $X$ is said to have {\color{blue}{\textsf{good reduction}}} if there exists a smooth and proper $R$-scheme $\mathscr{X}$ whose generic fiber $\mathscr{X}_K$ is isomorphic to $X$. In his 1967 paper \cite{ogg}, Ogg proved that an elliptic curve $E$ (a nice group variety of dimension $1$) defined over $K$ has good reduction if and only if the natural action of the inertia group $I_K$ on the $\ell$-adic Tate module of $E$ is trivial. This criterion (the {\textit{N\'{e}ron--Ogg--Shafarevich criterion}}) was later generalized by Serre and Tate \cite{serretate} to abelian varieties of arbitrary dimension. The nontriviality of the inertia action on the Tate module of an abelian variety is captured by the nonvanishing of a certain numerical invariant, called (the exponent of) the {\color{blue}{\textsf{conductor}}} of the abelian variety (see Section~\ref{defcond} for the definition). For an abelian variety defined over a number field, the local conductors at various primes appear in the conjectured functional equation for the $L$-function of the abelian variety.

Elliptic curves occupy a special place in the study of nice varieties, since they straddle the worlds of algebraic curves (nice varieties of dimension $1$) and abelian varieties (nice group varieties). One can ask if the N\'{e}ron-Ogg-Shafarevich criterion extends to curves of arbitrary genus, with $H^1(X_{\overline{K}},\Q_{\ell})$ in place of the the $\ell$-adic Tate module, but this turns out to be false \cite[Theorem~3.2]{oda} {\footnote{For an explicit example, see {\tt{\url{http://mathoverflow.net/questions/91909/a-curve-with-bad-reduction-for-which-the-jacobian-has-good-reduction}}}.}}.
% {\footnote{Let $E$ be an elliptic curve over $\Q_2$ such that $E$ has good reduction. Then the inertia action on $H^1(E_{\overline{\Q_2}},\Q_l)$ is trivial. Let $C$ be a nontrivial $E$-torsor. Then $C$ cannot have good reduction, since (i) the Hasse-Weil bounds imply that every smooth genus $1$ curve over $\F_2$ has a rational point, and, (ii) the valuative criterion for properness implies that any smooth $\F_2$-point on the reduction lifts to a smooth $\Q_2$-point on $C$. However, the $\Gal (\overline{\Q_2}/\Q_2)$-modules $H^1(E_{\overline{\Q_2}},\Q_l)$ and $H^1(C_{\overline{\Q_2}},\Q_l)$ are isomorphic. See \cite[Theorem~3.2]{oda} for an anabelian generalization of the N\'{e}ron-Ogg-Shafarevich criterion.}}
The correct generalization of the conductor also takes into account the dimensions of the cohomology groups of the special fiber of the minimal proper regular model of the algebraic curve, and this is encoded in another numerical invariant called the {\color{blue}{\textsf{Artin conductor}}} (see Section~\ref{defcond} for the definition). For curves of genus $g \geq 2$, the Artin conductor vanishes exactly when the minimal proper regular model of the curve is smooth over $R$. For a nice curve defined over a number field, the Artin conductors at various primes appear in the conjectured functional equation for the $L$-function associated to a global proper regular model of $X$ over the ring of integers of the number field \cite[Proposition 1.1]{bloch}. The (negative of the) Artin conductor is an upper bound for the conductor of the $I_K$-representation $H^1(X_{\overline{K}},\Q_{\ell})$, and the difference of the two conductors is one less than the number of the components in the special fiber of the minimal proper regular model of $X$ (Lemma~\ref{artcondfor}).

Elliptic curves over $K$ also have integral Weierstrass equations over $R$. An {\color{blue}{\textsf{integral Weierstrass equation}}} is an equation of the form 
\[ F(x,y,z) = y^2z+a_1xyz+a_3yz^2+x^3+a_2x^2z+a_4xz^2+a_6z^3 \]
that cuts out the elliptic curve in $\P^2_K$, with $\{a_1,a_2,\ldots,a_6\} \subset R$. 
% {\color{red}{We work with more general(even degree) integral Weierstrass equations for most of the paper --- should we give the definition that we use here (which might be confusing) or omit it altogether and refer to the Definitions section? Proposition 8(c) in \cite{liuminmod} explains why the two are equal... We could also make it a remark in the definitions section, after giving the more general definition for hyperelliptic curves.}} 
Any such equation has an associated (valuation of) {\color{blue}{\textsf{discriminant}}}{\footnote{For an elliptic curve over a global field, one can also define a global conductor and a global discriminant; our definitions of the discriminant and the conductor equal the valuation of the global discriminant and conductor; the definitions that we use are better suited to our situation since we are interested in studying the local behaviour at a {\textit{single}} prime.}} \cite[p.42, Section III.1]{babysil}, which is a nonnegative integer that measures how far the corresponding closed subscheme of $\P^2_R$ is from being smooth over $R$. An integral Weierstrass equation that minimizes the value of the discriminant is called a {\color{blue}{\textsf{minimal Weierstrass equation}}}, and the corresponding discriminant is called the {\color{blue}{\textsf{minimal discriminant}}}. It can be shown that an elliptic curve over $K$ has good reduction if and only if it has an integral Weierstrass equation with discriminant $0$ (Proposition~\ref{goodreddisccrit}\textup{(c)}). 

Since elliptic curves have these two measures of failure of good reduction, namely the Artin conductor and the minimal discriminant, it is quite natural to ask how these two invariants are related. In \cite{ogg}, Ogg showed that the minimal discriminant of an elliptic curve over $K$ equals the (negative) of the Artin conductor. He attributes this to a result of Tate from 1960 in the case when $\cha k \neq 2,3$, and remarks that the results in his paper are essentially about filling in the two remaining cases. However, the arguments in his paper fall short of handling the mixed characteristic $2$ case, i.e., when $\cha K = 0$ and $\cha k = 2$. The gap in his proof was finally filled in 20 years later by Saito in \cite{saito2}. The proof given by Ogg proceeds by a lengthy case by case analysis, and uses the Kodaira-N\'{e}ron classification of the possible special fibers of proper regular models of elliptic curves. 
%{\color{red}{Why does it fail in mixed char 2? Has something to do with Newton's method of solving polynomial equations...}} 
Saito's result, on the other hand, is far more general, and holds for proper regular models of {\textit{arbitrary}} curves, not just elliptic curves. 

To describe Saito's result, we first need to recall some earlier results of Mumford and Deligne. For any integer $g \geq 1$, let $\overline{\mathcal{M}}_g$ be the Deligne-Mumford compactification of the moduli space of smooth genus $g$ curves. In \cite[Theorem~5.10]{mumford}, Mumford established relations between certain natural classes of line bundles in $\Pic \overline{\mathcal{M}}_g$. Deligne used one of Mumford's relations as a template for defining a discriminant for an arbitrary proper regular model $\mathscr{X}$ over $\Spec R$ (see Section~\ref{defdeldisc} for the definition). Saito proved that the (negative of) the Artin conductor of a proper regular model $\mathscr{X}$ equals the discriminant that Deligne defined. This new discriminant, which we call the {\color{blue}{\textsf{Deligne discriminant}}}, coincides with the minimal discriminant in the case when $\mathscr{X}$ is the minimal proper regular model of an elliptic curve. In the case of genus $2$ curves, Saito relates his result to an explicit formula given by Ueno for the Deligne discriminant when $\cha k = 0$ or when $\cha k > 6$. Ueno's formula is in terms of yet another notion of discriminant that is special to genus $2$ curves, and it uses data pertaining to the geometry of the special fiber of a minimal proper regular model for a genus $2$ curve \cite{ueno}. The explicit classification that Ueno used for special fibers of proper regular models of genus $2$ curves already has over $120$ different types! For higher genus curves, the Deligne discriminant is very hard to explicitly compute in practice.

For a genus $g$ hyperelliptic curve, Liu defined a minimal discriminant that is analogous to the minimal discriminant for elliptic curves (see Section~\ref{mindisdef} for the definition). In \cite{liup}, Liu proved that for a genus $2$ hyperelliptic curve, the minimal discriminant is an upper bound for the (negative of the) Artin conductor. For genus $2$ curves, unlike elliptic curves, the minimal discriminant and the (negative of the) Artin conductor are sometimes different. In his paper, Liu gives an exact formula for the difference. When $\cha k \neq 2$, this difference can be computed quite explicitly in terms of the aforementioned classification of fibers of genus $2$ curves. When the hyperelliptic curve has semistable reduction over $K$, Kausz \cite{kausz} (when $\cha k \neq 2$) and Maugeais \cite{maugeais} (for all residue characteristics) prove theorems that relate the Deligne discriminant to yet another discriminant. We are not sure if Liu's minimal discriminant coincides with the discriminant that is used by Kausz and Maugeais. One of the main results in this thesis is the following.

\begin{thm}\label{main}
 Let $R$ be a discrete valuation ring with perfect residue field $k$. Assume that $\cha k \neq 2$. Let $K$ be the fraction field of $R$. Let $K^{\mathrm{sh}}$ denote the fraction field of the strict henselization of $R$. Let $C$ be a hyperelliptic curve over $K$ of genus $g$. Let $\nu \colon K \rightarrow \Z \cup \{\infty\}$ be the discrete valuation on $K$. Assume that the Weierstrass points of $C$ are $K^{\mathrm{sh}}$-rational. Let $S = \Spec R$ and let $\mathcal{X} \rightarrow S$ be the minimal proper regular model of $C$. Let $\nu(\Delta)$ denote the minimal discriminant of $C$. Then
 \[ -\Art (\mathcal{X}/S) \leq \nu(\Delta) .\]
\end{thm}

Our method of proof is different from Liu's method for genus $2$ curves. Liu compares the Deligne discriminant of the minimal proper regular model and the minimal discriminant by comparing both of them to a third  discriminant that he defines, following a definition given by Ueno \cite[Definition~1, Th\'{e}or\`{e}me~1 and Th\'{e}or\`{e}me~2]{liup}. This third discriminant is specific to genus $2$ curves.

We instead proceed by constructing an explicit proper regular model for $C$ (Section~\ref{construct}). We can immediately reduce to the case where $R$ is a henselian discrete valuation ring with algebraically closed residue field. We may then choose a minimal Weierstrass equation of the form, $y^2-f(x)$ where $f$ is a monic polynomial in $R[x]$ that splits completely. If the Weierstrass points of $C$ specialize to distinct points of the special fiber, then the polynomial $y^2-f(x)$, suitably homogenized, defines a smooth proper model of $C$ in a weighted projective space over $R$ (see Section~\ref{mindisdef} for the definition). If not, we iteratively blow up $\P^1_R$ until the Weierstrass points have distinct specializations. After a few additional blow-ups, we take the normalization of the resulting scheme in the function field of $C$. This gives us a (not necessarily minimal) proper regular model for $C$ (Theorem~\ref{Xisregular}). 

We have the relation $-\Art(X/S) = n(X_s)-1+\tilde{f}$ for a proper regular model $X$ of $C$, where $n(X_s)$ is the number of components of the special fiber of $X$ and $\tilde{f}$ is an integer that depends only on $C$ and not on the particular proper regular model chosen. This tells us that to bound $-\Art (X/S)$ for the minimal proper regular model from above, it suffices to bound $-\Art (X/S)$ for some proper regular model for the curve.

In Section~\ref{explicitformula}, we give an explicit formula for the Deligne discriminant for the model we have constructed. After a brief interlude on dual graphs in Section~\ref{dualgraphs}, we restate the formula for the Deligne discriminant using dual graphs. This formula tells us that the Deligne discriminant decomposes as a sum of terms, indexed by the vertices of the dual graph of the special fiber of the proper regular model we constructed (Section~\ref{DDanddual}). In Section~\ref{description}, we give a description of the rest of the strategy to prove the main theorem using this formula. The additional ingredients that are necessary are a decomposition of the minimal discriminant into a sum of terms depending on the local geometry of the graph (Section~\ref{breakupnaive}) and explicit formulae for the local terms in the Deligne discriminant in terms of dual graphs (Section~\ref{localcontribution}). In Section~\ref{comparison}, we show how to compare the Deligne discriminant and the minimal discriminant locally. To finish the proof, we sum the local inequalities to obtain $-\Art(X/S) \leq \nu(\Delta)$. 

One application of Theorem~\ref{main} is to give upper bounds on the number of components in the special fiber of the minimal proper regular model (Corollary~\ref{number}). This has applications to Chabauty's method of finding rational points on curves of genus at least $2$ \cite{poonenstoll}. 

It might be possible to adapt the same strategy to extend Theorem~\ref{main} to the case of nonrational Weierstrass points. The main difficulties in making this approach work are in understanding the right analogues of the results in Sections~\ref{breakupnaive} and \ref{localcontribution}.

In the second half of this thesis, we study the component group scheme attached to the special fiber of the N\'{e}ron model of a Jacobian. The N\'{e}ron model of an abelian variety $A$ defined over $K$ is in a certain sense the best possible extension of $A$ to a smooth, commutative group scheme $\mathcal{A}$ over $R$ (See Section~\ref{defneron} for a precise definition). The N\'{e}ron model $\mathcal{A}$ is proper over $S$ if and only if the abelian variety has good reduction. In general, the special fiber of the N\'{e}ron model might not be proper, or even connected. The special fiber of the N\'{e}ron model $\mathcal{A}_s$ fits in an exact sequence 
\[ 0 \rightarrow \mathcal{A}_s^0 \rightarrow \mathcal{A}_s \rightarrow \Phi \rightarrow 0,  \]
where $\mathcal{A}_s^0$ is a connected group scheme and $\Phi$ is a finite \'{e}tale group scheme, called the {\color{blue}{\textsf{component group scheme}}}. Let $\overline{k}$ be an algebraic closure of $k$. The {\color{blue}{\textsf{component group}}} is $\Phi(\overline{k})$, and the {\color{blue}{\textsf{Tamagawa number}}} is the order of the group $\Phi(k)$. 

Computing the orders of component groups and Tamagawa numbers have arithmetic applications. For an abelian variety $A$ over a number field $K$, bounds on the order of the component group at a prime is one of the inputs for giving bounds on the order of the torsion subgroup of $A(K)$. Mazur used this fact in his paper \cite{mazur} to prove that for an elliptic curve $E$ over $\Q$, the order of the torsion subgroup of $E(\Q)$ is bounded above by $16$. The local Tamagawa numbers appear in the statement of the full Birch and Swinnerton-Dyer conjecture; explicit verification of the full BSD conjecture for a specific abelian variety requires explicit computation of Tamagawa numbers. 

In the special case where the abelian variety is the Jacobian $J$ of a nice $K$-curve $X$, there are multiple approaches for constructing the N\'{e}ron model. Under relatively mild hypotheses, one can construct the N\'{e}ron model of $J$ by using the theory of the relative Picard scheme of a proper regular model of $X$ over $\Spec R$ \cite[Chapter~9, Section~5, Theorem~4]{blr}. This method also leads to a description of the component group of the special fiber of the N\'{e}ron model \cite[Chapter 9, Section 6, Theorem 1]{blr}: the component group is the middle homology group of a three term complex of free abelian groups, where the maps between the groups in the complex are given in terms of multiplicities and intersection numbers of the components in the special fiber of a proper regular model of $X \times K^{\textup{sh}}$ (see Section~\ref{defcompgrp} for details). The free abelian groups in this complex also admit natural actions of the absolute Galois group $G$ of $k$ that commute with the maps in the complex; Bosch and Liu used this action to give a description of $\Phi(k)$ similar to the description of the component group (Theorem~\ref{galcoh}), assuming that $G$ is procyclic.

The multiplicities and the intersection numbers of components in the special fiber of a proper regular model can be encoded in a weighted graph, called the {\color{blue}{\textsf{dual graph}}} of the special fiber (See Section~\ref{weightdualgraph} for the definition). In Theorem~\ref{compformula}, we give an explicit formula for the order of the component group that can be expressed in terms of the combinatorics of this dual graph, using the matrix-tree theorem (\ref{matrixtreeforgraphs}). Formulas of this type for the component group are not new, and special cases are well-known; when the proper regular model is semistable, the order of the component group equals the number of spanning trees in the dual graph, and when the dual graph is a tree, the order of the component group equals the product of the multiplicities of the components raised to certain exponents; each exponent depends on the number of neighbours of the corresponding vertex in the dual graph. The formula in Theorem~\ref{compformula} can be viewed as a hybrid of the formula in these two special cases, where each spanning tree in the dual graph is assigned a weight, and the expression for the weight is formally similar to the expression for the order of the component group in the case when the dual graph is a tree. A version of this formula is implicit in the proof of \cite[Corollary~3.5]{lor1}. Using a weighted version of the matrix-tree theorem (Theorem~\ref{matrixtree}), we obtain a similar formula for Tamagawa numbers expressed in terms of the combinatorics of a certain quotient graph.

For an elliptic curve having good or additive reduction the order of the component group is atmost $4$. Using our explicit formula in Theorem~\ref{compformula}, we prove an analogue of this fact for curves of higher genus (Theorem~\ref{uniformbound}). The condition of having good or additive reduction in genus $1$ is replaced by a certain geometric condition on the $-2$ curves in the minimal proper regular model in higher genus. To explain the nature of this geometric condition, we first recall that any connected commutative algebraic group over a perfect field admits a three step filtration, where the quotients are an abelian variety, a torus, and a unipotent group. The dimensions of the corresponding groups for the special fiber $\mathcal{A}_s^0$ of the N\'{e}ron model of any abelian variety, are called the abelian rank, toric rank and unipotent rank respectively. The condition of having good or additive reduction for an elliptic curve is equivalent to the elliptic curve having toric rank $0$. When the abelian variety is a Jacobian, the unipotent, abelian and toric ranks can be computed from the geometry of the special fiber of a proper regular model of the curve \cite[p.148]{lor}. The toric rank equals the first Betti number of the dual graph of the special fiber of a proper regular model. In \cite{lor}, Lorenzini shows that there is a uniform bound on the order of component groups of Jacobians having toric rank $0$. The condition that we impose is strictly weaker than requiring that $A$ have toric rank $0$, since we only disallow cycles of $-2$ curves, and not cycles of curves where at least one of the curves in the cycle has geometric genus $\geq 1$. 

Even though the order of the component group for a Jacobian is well-understood, the structure of the group remains quite mysterious. There are very few cases where the group structure is explicitly known; the only exceptions are genus $1$ curves where one can use the Kodaira-N\'{e}ron classification and a certain subset of Jacobians having potentially good reduction as described in \cite[Theorem~2.1]{lorpotgood} .
% Inspired by the explicit formula in Theorem~\ref{compformula} and the explicit calculations of component groups for elliptic curves for the various Kodaira-N\'{e}ron types, 
We prove a periodicity property of the component group under partial contraction of connecting chains (Theorem~\ref{compcontract}), generalizing \cite[Corollary~4.7]{bano} from the case of unweighted graphs. 

The formation of the N\'{e}ron model of an abelian variety does not commute with ramified base change: if $(R',K')$ is a ramified extension of $(R,K)$, the N\'{e}ron model of $A \times_K K'$ might not be equal to $\mathcal{A} \times_R R'$. The {\color{blue}{\textsf{N\'{e}ron component series}}} defined by Halle and Nicaise (see Section~\ref{defcompser} for the definition) simultaneously records the changes in the order of the component group under all tamely ramified extensions of the base in a power series. This power series is known to be rational in the following three cases: (i) when $A$ acquires semistable reduction after a tame extension of the base, (ii) when the toric rank equals the dimension of $A$ after a base extension, and (iii) when $A$ is a Jacobian. It is believed to be rational in general. Using our explicit formula (Theorem~\ref{compformula}), we provide an alternate proof of the key step in Halle and Nicaise's proof in case (iii), without having to resort to a reduction to the mixed characteristic case (Theorem~\ref{newproof}).



