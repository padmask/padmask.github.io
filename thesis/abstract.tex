% $Log: abstract.tex,v $
% Revision 1.1  93/05/14  14:56:25  starflt
% Initial revision
% 
% Revision 1.1  90/05/04  10:41:01  lwvanels
% Initial revision
% 
%
%% The text of your abstract and nothing else (other than comments) goes here.
%% It will be single-spaced and the rest of the text that is supposed to go on
%% the abstract page will be generated by the abstractpage environment.  This
%% file should be \input (not \include 'd) from cover.tex.
This thesis consists of two parts. In part one of this thesis, we study the relationship between the Artin conductor and the minimal discriminant of a hyperelliptic curve defined over the fraction field $K$ of a discrete valuation ring. The Artin conductor and the minimal discriminant are two measures of degeneracy of the singular fiber in a family of hyperelliptic curves. In the case of elliptic curves, the Ogg-Saito formula shows that (the negative of) the Artin conductor equals the minimal discriminant. In the case of genus $2$ curves, Liu showed that equality no longer holds in general, but the two invariants are related by an inequality. We extend Liu's inequality to hyperelliptic curves of arbitrary genus, assuming rationality of the Weierstrass points over $K$. 

In part two of this thesis, we compute the sizes of component groups and Tamagawa numbers of N\'{e}ron models of Jacobians using matrix tree theorems from combinatorics. Raynaud gave a description of the component group of the special fiber of the N\'{e}ron model of a Jacobian, in terms of the multiplicities and intersection numbers of components in the special fiber of a regular model of the underlying curve. Bosch and Liu used this description, along with some Galois cohomology computations to provide similar descriptions of Tamagawa numbers. We use various versions of the matrix tree theorem to make Raynaud's and Bosch and Liu's descriptions more explicit in terms of the combinatorics of the dual graph and the action of the absolute Galois group of the residue field on it. We then derive some consequences of these explicit descriptions. First, we use the explicit formula to provide a new geometric condition on the curve for obtaining a uniform bound on the size of the component group of its Jacobian. Then we prove a certain periodicity property of the component group of a Jacobian under contraction of connecting chains of specified lengths in the dual graph. As a third application, we obtain an alternate proof of one of the key steps in Halle and Nicaise's proof of the rationality of the N\'{e}ron component series for Jacobians. 